\documentclass{article}

\title{Statut Stowarzyszenia Hackerspace Pomorze}
\author{Zebranie Walne Stowarzyszenia Hackerspace Pomorze}
\date{5 marca 2022}

\usepackage[utf8]{inputenc}
\usepackage{polski}
\usepackage[polish]{babel}
\usepackage[margin=1in]{geometry}
\usepackage{needspace}
\usepackage[{center,nobottomtitles*}]{titlesec}
\usepackage{hyperref}

\renewcommand{\baselinestretch}{0} 
\renewcommand\thesection{Rozdział \Roman{section}.}
\newcommand{\myparagraph}[1]{\subsubsection*{#1}}

\begin{document}

\maketitle

\section{Postanowienia ogólne}
\myparagraph{§ 1}
\begin{enumerate}
\item
  Stowarzyszenie Hackerspace Pomorze zwane dalej Stowarzyszeniem, działa na podstawie Ustawy z dnia 7 kwietnia 1989 r. Prawo o stowarzyszeniach oraz postanowień niniejszego statutu.
\item
  Stowarzyszenie może używać nazwy skróconej Stowarzyszenie HS Pomorze.
\end{enumerate}

\myparagraph{§ 2}
Siedzibą Stowarzyszenia jest miasto Gdańsk.

\myparagraph{§ 3}
\begin{enumerate}
\item
  Terenem działania Stowarzyszenia jest obszar Rzeczpospolitej Polskiej.
\item
  Dla realizacji celów statutowych Stowarzyszenie może prowadzić działania poza granicami kraju na terenie innych państw, z poszanowaniem tamtejszego prawa.
\end{enumerate}

\myparagraph{§ 4}
Stowarzyszenie posiada osobowość prawną. Powołane jest na czas nieokreślony.

\myparagraph{§ 5}
Stowarzyszenie współpracuje z krajowymi, zagranicznymi i międzynarodowymi organizacjami pozarządowymi i innymi instytucjami. Może pozostawać członkiem tych organizacji na zasadach pełnej autonomii.

\myparagraph{§ 6}
Stowarzyszenie opiera swoją działalność na pracy społecznej członków. Do prowadzenia swoich działań może zatrudniać pracowników oraz powoływać biura.

\myparagraph{§ 7}
Członkami Stowarzyszenia mogą być cudzoziemcy, włącznie z osobami nie mającymi miejsca zamieszkania na terytorium Rzeczypospolitej Polskiej.

\section{Cele i sposoby ich realizacji}
\myparagraph{§ 8}
Celem Stowarzyszenia jest:
\begin{enumerate}
\item
  Działalność edukacyjna, naukowa i oświatowo-wychowawcza, wspieranie idei nauki otwartej wobec obywateli i przez obywateli tworzonej.
\item
  Popularyzacja nauki i metody naukowej.
\item
  Wspieranie postaw obywatelskich, w szczególności wśród młodzieży.
\item
  Upowszechnianie i ochrona praw człowieka oraz swobód obywatelskich.
\item
  Upowszechnianie i ochrona wolności osobistej i gospodarczej.
\item
  Promowanie wolności w zakresach: światopoglądu, wolności do wyboru alternatywnej edukacji, prowadzenia działalności gospodarczej,
  wolności informacji, wolnego i otwartego oprogramowania, otwartych technologii (ang. open hardware).
\item
  Podejmowanie działań na rzecz pomocy, rozwoju i aktywizacji oraz wspierania postaw przedsiębiorczych, w szczególności wśród młodzieży, osób niepełnosprawnych i wykluczonych zawodowo.
\item
  Wspieranie aktywności społecznej i kulturalnej, w szczególności wśród młodzieży.
\item
  Upowszechnianie kultury fizycznej, sportu i turystyki;
\end{enumerate}

\myparagraph{§ 9}
Stowarzyszenie realizuje swoje cele poprzez:
\begin{enumerate}
\item
  organizowanie kółek zainteresowań obejmujących konkretną tematykę zwanych dalej wydziałami,
\item
  organizowanie wykładów, szkoleń, konferencji, spotkań dyskusyjnych, imprez kulturalnych, konwentów, spotkań tematycznych, debat, imprez kulturalnych umożliwiających artystom udział we wszelkich formach rozwoju artystycznego, obozów, kolonii, seminariów, kursów, prelekcji, warsztatów, innych form czynnego wypoczynku i imprez sportowo-rekreacyjnych w ramach zainteresowań konkretnych wydziałów,
\item
  organizacja i najem przestrzeni społecznych do dzielenia się wiedza i wspólnej nauki znanych powszechnie jako Makerspace, Hackerspace, DIY Biolab, pracownie artystyczne, sale szkoleniowe, sale dydaktyczne czy inne przestrzenie wspierające rozwój, naukę i kulturę.
\item
  nawiązywanie kontaktów z firmami, organizacjami pokrewnymi ideowo i instytucjami w Polsce i na terenie obcych krajów ku współpracy dla
  realizacji celów statutowych oraz pozyskiwaniu patronów, darczyńców, partnerów, ekspertów, prelegentów,
\item
  umożliwienie młodzieży dodatkowej edukacji i rozwoju poza szkołą, a rodzicom wybierającym edukację domową współtworzenie i kreowanie we własnym zakresie formy i treści nauczania na zajęciach organizowanych przy współpracy ze stowarzyszeniem, korzystając z wydziałów stowarzyszenia,
\item
  umożliwienie osobom dorosłym kontynuowania swojej osobistej edukacji we własnym zakresie poprzez działalność w wydziałach,
\item
  zapewniania w ramach wydziałów edukacji i rozwoju naukowego, badawczego, kulturowego, wychowawczego, psychicznego i fizycznego zdolnego stanowić alternatywę lub uzupełnienie dla sformalizowanego systemu edukacji poprzez prowadzenie działalności wydawniczej i organizowanie wydarzeń, tworząc przez to środowisko sprzyjające rozwojowi,
\item
  prowadzenie działalności wydawniczej oraz dystrybucja publikacji,
\item
  organizowanie wyjazdów, wymian, imprez kulturalnych i naukowych, spotkań międzynarodowych i międzykulturowych, fundowanie stypendiów oraz wspieranie niematerialne osób w trudnej sytuacji życiowej, szczególnie zdolnej młodzieży, studentów i dorosłych pragnących rozwijać się w citizen science poprzez udział w rozwoju nauki oraz osób niepełnosprawnych lub wykluczonych społecznie i zawodowo,
\item
  prowadzenie nieodpłatnej i odpłatnej działalności pożytku publicznego dla realizacji zadań statutowych stowarzyszenia,
\item
  prowadzenie działalności badawczej oraz sporządzanie analiz, ekspertyz, projektów ustaw, raportów i opinii zawierających propozycje rozwiązań w obszarach społecznym, politycznym i gospodarczym służących rozwojowi Rzeczypospolitej Polskiej.
\end{enumerate}

\myparagraph{§ 9a}
Wszelkie utwory powstałe w wyniku działalności statutowej Stowarzyszenia lub na potrzeby realizowania działalności statutowej Stowarzyszenia muszą zostać udostępnione na wolnej licencji (free license), znajdującej się na liście utrzymywanej przez Free Software Foundation pod adresem \url{https://www.gnu.org/licenses/license-list.html}

\section{Członkowie Stowarzyszenia}
\myparagraph{§ 10}
\begin{enumerate}
\item
  Członkami Stowarzyszenia mogą być osoby fizyczne i prawne. Osoba prawna może być jedynie członkiem wspierającym Stowarzyszenia.
\item
  Stowarzyszenie posiada członków:
  \begin{enumerate}
  \def\labelenumii{\alph{enumii}.}
  \item
    zwyczajnych,
  \item
    nadzwyczajnych
  \item
    wspierających,
  \item
    honorowych.
  \end{enumerate}
\end{enumerate}

\myparagraph{§ 11}
\begin{enumerate}
\item
  Członkiem zwyczajnym Stowarzyszenia może być każda osoba fizyczna, która złoży pisemną deklarację zawierającą rekomendację trzech członków Stowarzyszenia.
\item
  Przyjęcia nowych członków dokonuje Zarząd uchwałą podjętą nie później niż w ciągu dwóch miesięcy od daty złożenia deklaracji.
\item
  Członkiem nadzwyczajnym Stowarzyszenia może być każda osoba fizyczna, która złoży pisemną deklarację zawierającą rekomendację trzech członków Stowarzyszenia.
\item
  Przyjęcia nowych członków dokonuje Zarząd uchwałą podjętą nie później niż w ciągu dwóch miesięcy od daty złożenia deklaracji.
\item
  Członkiem wspierającym Stowarzyszenia może zostać osoba fizyczna lub prawna deklarująca pomoc finansową, rzeczową lub merytoryczną w realizacji celów Stowarzyszenia.
\item
  Członkiem wspierającym staje się po złożeniu pisemnej deklaracji na podstawie uchwały Zarządu podjętej nie później niż w ciągu dwóch miesięcy od daty złożenia deklaracji.
\item
  Członkiem honorowym Stowarzyszenia może być osoba fizyczna, która wniosła wybitny wkład w działalność i rozwój Stowarzyszenia.
\item
  Członkiem honorowym staje się po przyjęciu uchwały przez Walne Zebranie na wniosek Zarządu albo co najmniej 6 członków Stowarzyszenia.
\end{enumerate}

\myparagraph{§ 12.Z.1}
Członkowie zwyczajni mają prawo:
\begin{enumerate}
\item
  biernego i czynnego uczestniczenia w wyborach do władz Stowarzyszenia,
\item
  korzystania z dorobku i wszelkich form działalności Stowarzyszenia,
\item
  udziału w zebraniach oraz imprezach organizowanych przez Stowarzyszenie,
\item
  zgłaszania wniosków co do działalności Stowarzyszenia.
\end{enumerate}

\myparagraph{§ 12.Z.2}
Członkowie zwyczajni mają obowiązek:
\begin{enumerate}
\item
  brania udziału w działalności Stowarzyszenia i w realizacji jego celów,
\item
  uczestniczenia w walnych zebraniach członków,
\item
  przestrzegania statutu i uchwał władz Stowarzyszenia,
\item
  regularnego opłacania składek.
\end{enumerate}

\myparagraph{§ 12.NZ.1}
\begin{enumerate}
\item
  Prawa i obowiązki każdego członka nadzwyczajnego z osobna definiuje zarząd.
\item
  Prawa i obowiązki członka nadzwyczajnego mogą być zmienione na jego wniosek po zatwierdzeniu przez zarząd.
\item
  Prawa członka nadzwyczajnego muszą zawierać się w zbiorze praw członka zwyczajnego.
\item
  Członek nadzwyczajny musi mieć obowiązek regularnego opłacania składek członkowskich.
\item
  Członek nadzwyczajny musi mieć obowiązek przestrzegania statutu i uchwał władz stowarzyszenia.
\item 
  Członek nadzwyczajny musi nie posiadać biernego oraz czynnego prawa wyborczego.
\item
  Członek zwykły może w każdym momencie zostać członkiem nadzwyczajnym informując w formie pisemnej zarząd.
\end{enumerate}

\myparagraph{§ 12.W.1}
\begin{enumerate}
\item
  Członkowie wspierający nie posiadają biernego oraz czynnego prawa wyborczego, mogą jednak brać udział z głosem doradczym w statutowych władzach Stowarzyszenia, poza tym posiadają takie prawa jak członkowie zwyczajni.
\item
  Członek wspierający ma obowiązek wywiązywania się z zadeklarowanych świadczeń, przestrzegania statutu oraz uchwał władz Stowarzyszenia.
\item
  Członek wspierający musi opłacać składki członkowskie.
\item
  Członek zwykły może w każdym momencie zostać członkiem wspierającym informując w formie pisemnej zarząd.
\end{enumerate}

\myparagraph{§ 12.H.1}
\begin{enumerate}
\item
  Członkowie honorowi nie posiadają biernego oraz czynnego prawa wyborczego, mogą jednak brać udział z głosem doradczym w statutowych władzach Stowarzyszenia, poza tym posiadają takie prawa jak członkowie zwyczajni.
\item
  Członek honorowy ma obowiązek przestrzegania statutu oraz uchwał władz Stowarzyszenia. 
\item
  Członek honorowy posiada dobrowolność w opłacaniu składek członkowskich.
\end{enumerate}


\myparagraph{§ 15}
\begin{enumerate}
\item
  Członkostwo w Stowarzyszeniu ustaje na skutek:
  \begin{enumerate}
  \def\labelenumii{\alph{enumii}.}
  \item
    dobrowolnej rezygnacji pisemnej z przynależności do Stowarzyszenia złożonej do Zarządu,
  \item
    wykluczenia przez Zarząd z powodu:
    \begin{enumerate}
    \def\labelenumiii{\roman{enumiii}.}
    \item
      nieusprawiedliwionego zalegania z opłatą składek członkowskich lub innych zobowiązań, przez okres przekraczający sześć miesięcy,
    \item
      naruszenia zasad statutu, nieprzestrzegania postanowień i uchwał władz Stowarzyszenia,
    \item
      braku przejawów aktywnej działalności na rzecz Stowarzyszenia.
    \end{enumerate}
  \item
    utraty praw obywatelskich na mocy prawomocnego wyroku sądu,
  \item
    śmierci członka lub utraty osobowości prawnej przez członka
    wspierającego.
  \end{enumerate}
\item
  Od uchwały Zarządu w sprawie pozbawienia członkostwa w Stowarzyszeniu przysługuje odwołanie do Walnego Zebrania Członków w terminie 14 dni od daty doręczenia stosownej uchwały. Odwołanie jest rozpatrywane na najbliższym Walnym Zebraniu Członków. Uchwała Walnego Zebrania jest ostateczna.
\end{enumerate}

\section{Władze Stowarzyszenia}
\myparagraph{§ 16}
Władzami Stowarzyszenia są:
\begin{enumerate}
\item
  Walne Zebranie Członków,
\item
  Zarząd,
\item
  Komisja Rewizyjna.
\end{enumerate}

\myparagraph{§ 17}
W razie, gdy skład władz wybieralnych Stowarzyszenia ulegnie zmniejszeniu w czasie trwania kadencji uzupełnienie ich składu może nastąpić w drodze kooptacji której dokonują pozostali członkowie organu, który uległ zmniejszeniu. Liczba członków Stowarzyszenia dokooptowanych do zarządu oraz komisji rewizyjnej nie może przekroczyć połowy liczby członków pochodzących z wyboru.

\myparagraph{§ 18}
Kadencja wszystkich wybieralnych władz Stowarzyszenia trwa 2 lata.

\myparagraph{§ 19}
\begin{enumerate}
\item
  Uchwały wszystkich organów Stowarzyszenia zapadają zwykłą większością głosów przy obecności co najmniej połowy członków uprawnionych do głosowania, chyba że dalsze postanowienia statutu stanowią inaczej.
\item
  Uchwały o wyborze i odwoływaniu organów stowarzyszenia, rozwiązaniu stowarzyszenia oraz zmianach statutu zapadają większością 2/3 głosów, przy obecności co najmniej połowy członków uprawnionych do głosowania.
\end{enumerate}

\subsection*{Walne Zebranie Członków}
\myparagraph{§ 20}
\begin{enumerate}
\item
  Najwyższą władzą Stowarzyszenia jest Walne Zebranie Członków.
\item
  Walne Zebranie może być zwołane w trybie zwyczajnym i nadzwyczajnym.
\item
  Dla zwołania Walnego Zebrania Członków Zarząd podaje dwa terminy. Jeśli w pierwszym terminie nie zbierze się połowa członków zwyczajnych, to Walne Zebranie Członków odbywa się w drugim terminie.
\item
  Walne Zebranie w trybie zwyczajnym zwołuje Zarząd raz w roku, jako sprawozdawcze, i co dwa lata, jako sprawozdawczo-wyborcze, zawiadamiając członków o jego terminie i miejscu za pomocą poczty elektronicznej, co najmniej na 14 dni przed pierwszym terminem Walnego Zebrania. Zawiadomienie musi zawierać sprawozdanie z działalności Stowarzyszenia oraz projekt porządku obrad. Oba terminy Walnego Zebrania Członków w trybie zwyczajnym muszą być odległe od siebie przynajmniej o 7, ale nie bardziej niż o 14 dni kalendarzowych.
\item
  Walne Zebranie obraduje wg uchwalonego przez siebie regulaminu obrad.
\item
  Walne Zebranie w trybie nadzwyczajnym zwołuje Zarząd:
  \begin{enumerate}
  \def\labelenumii{\alph{enumii}.}
  \item
    z własnej inicjatywy,
  \item
    na żądanie Komisji Rewizyjnej,
  \item
    na pisemny wniosek co najmniej 1/3 ogólnej liczby członków zwyczajnych Stowarzyszenia.
  \end{enumerate}
\item
  Nadzwyczajne Walne Zebranie powinno zostać zwołane przed upływem 21 dni od daty zgłoszenia wniosku lub żądania i obradować nad sprawami, dla których zostało zwołane.
\item
  Dla zwołania Walnego Zebrania Członków w trybie nadzwyczajnym Zarząd podaje termin obrad do wiadomości wszystkich członków pocztą elektroniczną co najmniej 7 dni przed pierwszym terminem zebrania.
\item
  Oba terminy Walnego Zebrania Członków w trybie nadzwyczajnym muszą być odległe od siebie przynajmniej o 3, ale nie bardziej niż o 7 dni kalendarzowych.
\item
  Na żądanie co najmniej 1⁄4 liczby członków Stowarzyszenia obecnych na walnym zebraniu przewodniczący zarządza głosowanie tajne w innych sprawach -- objętych porządkiem obrad, jak też zgłoszonym na zebraniu.
\end{enumerate}

\myparagraph{§ 21}
Do kompetencji Walnego Zebrania należy:
\begin{enumerate}
\item
  określenie głównych kierunków działania i rozwoju Stowarzyszenia,
\item
  uchwalanie zmian statutu,
\item
  wybór i odwoływanie wszystkich władz Stowarzyszenia,
\item
  udzielanie Zarządowi absolutorium na wniosek Komisji Rewizyjnej,
\item
  rozpatrywanie i zatwierdzanie sprawozdań władz Stowarzyszenia,
\item
  rozpatrywanie odwołań od uchwał Zarządu,
\item
  podejmowanie uchwały o rozwiązaniu Stowarzyszenia i przeznaczeniu jego majątku,
\item
  podejmowanie uchwał w sprawie przyjęcia członka honorowego.
\end{enumerate}

\subsection*{Zarząd}
\myparagraph{§ 22}
\begin{enumerate}
\item
  Zarząd jest powołany do kierowania całą działalnością Stowarzyszenia zgodnie z uchwałami Walnego Zebrania Członków, reprezentuje Stowarzyszenie na zewnątrz.
\item
  Zarząd składa się z minimum 3 osób, spośród których na pierwszym posiedzeniu wybiera prezesa, jednego lub więcej wiceprezesa i skarbnika.
\item
  Posiedzenia Zarządu odbywają się w miarę potrzeb, nie rzadziej jednak niż raz na kwartał. Posiedzenia Zarządu zwołuje prezes.
\end{enumerate}

\myparagraph{§ 23}
Do kompetencji Zarządu należy:
\begin{enumerate}
\item
  kierowanie bieżącą pracą Stowarzyszenia,
\item
  realizowanie uchwał Walnego Zebrania,
\item
  zarządzanie majątkiem Stowarzyszenia,
\item
  planowanie i prowadzenie gospodarki finansowej,
\item
  reprezentowanie Stowarzyszenia na zewnątrz i działanie w jego imieniu,
\item
  przyjmowanie i wykluczanie członków Stowarzyszenia,
\item
  zwoływanie Walnego Zebrania,
\item
  ustalanie wysokości składek członkowskich,
\item
  wybór i odwołanie przewodniczącego wydziału,
\item
  możliwość powoływania pełnomocników do każdej podejmowanej przez stowarzyszenie sprawy.
\end{enumerate}

\myparagraph{§ 24}
\begin{enumerate}
\item
  Uchwały zarządu podejmowane są w głosowaniu jawnym, zwykłą większością głosów, przy obecności co najmniej połowy ogólnej liczby uprawnionych
  członków (quorum).
\item
  W sytuacji równego rozłożenia głosów decyduje głos Prezesa.
\item
  W przypadkach nie cierpiących zwłoki głosowanie może odbyć się w drodze ustalenia telefonicznego, za pomocą poczty elektronicznej bądź ustalenia korespondencyjnego, z dopełnieniem staranności powiadomienia wszystkich członków zarządu.
\end{enumerate}

\subsection*{Komisja Rewizyjna}
\myparagraph{§ 25}
\begin{enumerate}
\item
  Komisja Rewizyjna powołana jest do sprawowania kontroli nad działalnością Stowarzyszenia.
\item
  Komisja Rewizyjna składa się z minimum 3 osób, w tym Przewodniczącego wybieranego na pierwszym posiedzeniu komisji.
\item
  Posiedzenia Komisji Rewizyjnej odbywają się w miarę potrzeb nie rzadziej jednak niż raz w roku. Posiedzenia Komisji zwołuje Przewodniczący.
\item
  Członkowie Komisji Rewizyjnej mogą być odwołani na podstawie uchwały walnego zebrania członków Stowarzyszenia w przypadku, gdy:
  \begin{enumerate}
  \def\labelenumii{\alph{enumii}.}
  \item
    nie wywiązuje się z obowiązków członka Stowarzyszenia i komisji,
  \item
    działają na szkodę Stowarzyszenia.
  \end{enumerate}
\item
  Uchwały Komisji Rewizyjnej są ważne, jeżeli zostały podjęte w obecności większości członków.
\item
  Uchwały Komisji Rewizyjnej zapadają zwykłą większością głosów. W razie równej ilości głosów decyduje głos Przewodniczącego. Głosowania są jawne.
\item
  Do współpracy i opracowania określonych zagadnień Komisja Rewizyjna może powołać, za zgodą prezesa zarządu rzeczoznawców lub specjalistów.
\end{enumerate}

\myparagraph{§ 26}
Do kompetencji Komisji Rewizyjnej należy:
\begin{enumerate}
\item
  kontrola całokształtu działalności Stowarzyszenia,
\item
  przeprowadzenie merytorycznej i finansowej kontroli działalności zarządu -- co najmniej raz w roku,
\item
  składanie sprawozdań na Walnym Zebraniu Członków wraz z oceną działalności Stowarzyszenia i Zarządu Stowarzyszenia,
\item
  wnioskowanie do walnego zebrania członków o udzielanie absolutorium Zarządowi,
\item
  wnioskowanie o odwołanie Zarządu lub poszczególnych członków Zarządu w razie jego bezczynności,
\item
  wnioskowanie o zwołanie Nadzwyczajnego Walnego Zebrania Członków.
\end{enumerate}

\section{Majątek i gospodarka finansowa}
\myparagraph{§ 27}
\begin{enumerate}
\item
  Źródłami powstania majątku Stowarzyszenia są:
  \begin{enumerate}
  \def\labelenumii{\alph{enumii}.}
  \item
    składki członkowskie,
  \item
    darowizny, zapisy i spadki, środki pochodzące z ofiarności publicznej
  \item
    dotacje, subwencje, udziały, odsetki od środków zgromadzonych na kontach bankowych.
  \end{enumerate}
\item
  Stowarzyszenie prowadzi gospodarkę finansową zgodnie z obowiązującymi przepisami.
\item
  Stowarzyszenie prowadzi działalność pożytku publicznego bezpłatną i odpłatną.
\item
  Decyzje w sprawie nabywania, zbywania i obciążania majątku Stowarzyszenia podejmuje Zarząd.
\end{enumerate}

\myparagraph{§ 28}
Do składania oświadczeń woli w imieniu Stowarzyszenia, w tym w sprawach majątkowych, uprawnionych jest dwóch członków zarządu działających łącznie.

\section{Postanowienia końcowe}
\myparagraph{§ 29}
W przypadku podjęcia uchwały o rozwiązaniu Stowarzyszenia Walne Zebranie Członków określa sposób jego likwidacji oraz przeznaczenie majątku Stowarzyszenia.


\end{document}
